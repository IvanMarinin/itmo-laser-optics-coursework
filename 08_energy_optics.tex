%Энергетический расчет с оценкой плотности энергии/мощности на: оптических элементах (зеркалах и линзах); в промежуточных действительных перетяжках; в крайней перетяжке, предназначенной для обработки материала.

\chapter{Энергетический расчет в оптических элементах}

\section{Исходные данные для расчета}

Для энергетического расчета используются следующие параметры лазерного излучения и оптической системы:

\begin{itemize}
    \item Мощность лазера $P = 12$ Вт
    \item Длительность импульса: $\tau = 150$ мкс
    \item Частота повторения импульсов: $f = 20$ кГц
    \item Диаметр пучка на выходе лазера: 3.5 мм
    \item Телескоп: увеличение 10x, пропускание 94\%
    \item Зеркала: 4 шт., отражение каждого 99.7\%
    \item Фокусирующая линза: пропускание 95\%
\end{itemize}

\section{Расчет потерь в оптической системе}

Суммарные потери в оптической системе составляют:

\[
\Phi_{\text{loss}} = 1 - (0.94 \cdot 0.997^4 \cdot 0.95) = 
\]

Энергия в импульсе после оптической системы:

\[
E_{\text{out}} = E_p \cdot (1 - \Phi_{\text{loss}}) = 0.6 \cdot 0.881 = 0.5286 \, \text{мДж}
\]

\section{Плотность энергии на оптических элементах}

\subsection{Входная линза телескопа}

\begin{itemize}
    \item Диаметр пучка: 3.5 мм
    \item Площадь пятна: $A = \pi \cdot (1.75)^2 = 9.62 \, \text{мм}^2 = 0.0962 \, \text{см}^2$
    \item Плотность энергии: $\varepsilon = \dfrac{E_p}{A} = \dfrac{0.6}{0.0962} = 6.24 \, \text{мДж/см}^2$
\end{itemize}

\subsection{Выходная линза телескопа}

\begin{itemize}
    \item Диаметр пучка: 17.5 мм
    \item Площадь пятна: $A = \pi \cdot (8.75)^2 = 240.5 \, \text{мм}^2 = 2.405 \, \text{см}^2$
    \item Энергия после телескопа: $E = 0.6 \cdot 0.94 = 0.564 \, \text{мДж}$
    \item Плотность энергии: $\varepsilon = \dfrac{0.564}{2.405} = 0.234 \, \text{мДж/см}^2$
\end{itemize}

\subsection{Первое зеркало (M1)}

\begin{itemize}
    \item Диаметр пучка: 17.5 мм
    \item Площадь пятна: 2.405 см²
    \item Энергия на зеркале: 0.564 мДж
    \item Плотность энергии: $\varepsilon = 0.234 \, \text{мДж/см}^2$
\end{itemize}

\subsection{Фокусирующая линза}

\begin{itemize}
    \item Диаметр пучка: 16.89 мм
    \item Площадь пятна: $A = \pi \cdot (8.445)^2 = 224.0 \, \text{мм}^2 = 2.24 \, \text{см}^2$
    \item Энергия на линзе: $E = 0.6 \cdot 0.94 \cdot 0.997^3 = 0.559 \, \text{мДж}$
    \item Плотность энергии: $\varepsilon = \dfrac{0.559}{2.24} = 0.250 \, \text{мДж/см}^2$
\end{itemize}

\section{Плотность энергии в фокальном пятне}

\subsection{Расчет параметров}

\begin{itemize}
    \item Диаметр фокального пятна: 60.5 мкм
    \item Площадь пятна: $A = \pi \cdot (30.25 \cdot 10^{-3})^2 = 2.87 \cdot 10^{-3} \, \text{мм}^2 = 2.87 \cdot 10^{-5} \, \text{см}^2$
    \item Энергия в импульсе: 0.5286 мДж
    \item Плотность энергии: $\varepsilon = \dfrac{0.5286}{2.87 \cdot 10^{-5}} = 18420 \, \text{мДж/см}^2 = 18.42 \, \text{Дж/см}^2$
\end{itemize}

\subsection{Плотность мощности}

\[
q = \frac{\varepsilon}{\tau} = \frac{18.42}{150 \cdot 10^{-6}} = 122.8 \, \text{кВт/см}^2
\]

\section{Сравнение с пороговыми значениями}

\subsection{Порог обработки материала}

Согласно расчетам в разделе энергетических требований, для лазерной обработки стекла требуется плотность мощности не менее 31.6 кВт/см². Полученное значение 122.8 кВт/см² превышает пороговое значение в 3.9 раза, что обеспечивает надежную обработку материала.

\subsection{Лучевая прочность оптических элементов}

\begin{itemize}
    \item Телескоп GBE05-E3: порог повреждения 10 Дж/см² (для импульсов 100 нс)
    \item Зеркала: порог повреждения 5 Дж/см²
    \item Фокусирующая линза: порог повреждения 10 Дж/см²
\end{itemize}

Фактические плотности энергии на оптических элементах:
\begin{itemize}
    \item Входная линза телескопа: 6.24 мДж/см²
    \item Выходная линза телескопа: 0.234 мДж/см²
    \item Зеркало M1: 0.234 мДж/см²
    \item Фокусирующая линза: 0.250 мДж/см²
\end{itemize}

Все значения значительно ниже порогов повреждения, что гарантирует безопасную эксплуатацию оптической системы.

\section{Выводы}

\begin{itemize}
    \item Оптическая система обеспечивает достаточную плотность энергии в фокальном пятне для надежной обработки материалов
    \item Плотности энергии на оптических элементах не превышают пороги повреждения
    \item Имеется значительный запас по энергии для компенсации возможных потерь
    \item Система обладает высокой надежностью и безопасностью в эксплуатации
\end{itemize}
\endinput