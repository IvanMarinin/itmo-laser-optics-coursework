% С учетом заданных параметров дать оценку динамических характеристик лазерного пучка в плоскости обработки (определить: скорость, ускорение, мощность двигателя). Описать движение лазерного пучка в плоскости обработки.


\chapter{Динамические характеристики системы сканирования}

\section{Расчет параметров сканирования}

\subsection{Шаг перемещения по осям XY}

Для обеспечения перекрытия пятен на 80\% по диаметру, шаг перемещения между импульсами составляет:

\[
d_{xy} = (1 - 0.8) \cdot d = 0.2 \cdot 60 = 12 \, \text{мкм}
\]

где $d = 60$ мкм -- диаметр фокального пятна.

\subsection{Минимальная скорость сканирования}

При частоте повторения импульсов $f_{\text{имп}} = 100$ кГц, минимальная скорость сканирования, обеспечивающая заданное перекрытие:

\[
v_{\text{min}} = d_{xy} \cdot f_{\text{имп}} = 12 \cdot 10^{-6} \cdot 100 \cdot 10^{3} = 1.2 \, \text{м/с}
\]

\subsection{Шаг перемещения по оси Z}

Глубина резкости определяется длиной Рэлея. Радиус перетяжки пучка:

\[
\omega_{02} = \frac{d}{2} = 30 \, \text{мкм}
\]

Длина Рэлея:

\[
z_R = \frac{\pi \omega_{02}^2}{M^2 \lambda} = \frac{\pi \cdot (30 \cdot 10^{-6})^2}{1.1 \cdot 10.6 \cdot 10^{-6}} \approx 0.242 \, \text{мм}
\]

Шаг перемещения по оси Z не должен превышать $0.2 z_R$:

\[
\text{step}_z = 0.2 \cdot z_R = 0.2 \cdot 0.242 = 0.0484 \, \text{мм} = 48.4 \, \text{мкм}
\]

\section{Динамические параметры системы}

\subsection{Расчет пути разгона и торможения}

Для расчета пути разгона используем энергетический подход. Кинетическая энергия каретки при скорости $v = 0.4$ м/с:

\[
E_{\text{кин}} = \frac{m v^2}{2} = \frac{1.2 \cdot (0.4)^2}{2} = 0.096 \, \text{Дж}
\]

При мощности двигателя $P = 550$ Вт, время разгона до заданной скорости:

\[
t = \frac{E_{\text{кин}}}{P} = \frac{0.096}{550} \approx 0.0001745 \, \text{с}
\]

Ускорение каретки:

\[
a = \frac{v}{t} = \frac{0.4}{0.0001745} \approx 2292 \, \text{м/с}^2
\]

Путь разгона:

\[
s = \frac{a t^2}{2} = \frac{2292 \cdot (0.0001745)^2}{2} \approx 0.0000349 \, \text{м} = 0.0349 \, \text{мм}
\]

\subsection{Анализ реалистичности расчетов}

Полученные значения (ускорение 2292 м/с², путь разгона 0.035 мм) являются теоретическим пределом и не достижимы на практике. В реальных портальных системах ускорения составляют 5-30 м/с².

\subsection{Итоговые динамические характеристики}

\begin{itemize}
    \item Минимальная скорость сканирования: $1.2$ м/с
    \item Рабочая скорость: $0.4$ м/с
    \item Шаг по XY: $12$ мкм
    \item Шаг по Z: $48.4$ мкм  
    \item Теоретический путь разгона: $0.035$ мм
    \item Реалистичное время разгона: $40$ мс
\end{itemize}

\section{Описание движения лазерного пучка}

Движение фокусирующей оптики осуществляется по заданным траекториям с синхронизацией импульсов лазера. Система обеспечивает:

\begin{itemize}
    \item Высокую скорость позиционирования до 1.2 м/с
    \item Точное перемещение с шагом 12 мкм по осям XY
    \item Автофокусировку с шагом 48.4 мкм по оси Z
    \item Перекрытие не менее 80\% при рабочей скорости
    \item Быстрое ускорение благодаря запасу мощности двигателя
\end{itemize}

Мощность двигателя 550 Вт обеспечивает значительный запас для преодоления сил трения и инерции, позволяя реализовать сложные траектории движения с частыми изменениями направления.
\endinput