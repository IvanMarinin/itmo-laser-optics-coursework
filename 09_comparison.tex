% Сравнение плотности энергии/мощности с лучевой прочностью зеркал и линз, а также с порогами лазерной обработки материала.

\chapter{Сравнение с прочностными пределами}

Проведем сравнение рассчитанных параметров с повреждающими порогами оптических элементов. Используем золотые зеркала Thorlabs PFE10-M01~\cite{Miror}.

\section{Сравнительные данные}

\begin{table}[H]
\caption{Сравнение с повреждающими порогами}
\centering
\begin{tabular}{lccc}
\hline
Параметр & Наши значения & Пороги элементов & Запас порядка \\
\hline
\textbf{Расширитель пучка} & & & \\
Плотность энергии & 3.93 мДж/см$^2$ & 15.8 Дж/см$^2$ & $10^3$ \\
Плотность мощности & 3.93 кВт/см$^2$ & 15.8 МВт/см$^2$ & $10^3$ \\
\hline
\textbf{Зеркало} & & & \\
Плотность энергии & 0.155 мДж/см$^2$ & 20.0 Дж/см$^2$ & $10^5$ \\
Плотность мощности & 0.155 кВт/см$^2$ & 20.0 МВт/см$^2$ & $10^5$ \\
\hline
\textbf{Фокусирующая линза} & & & \\
Плотность энергии & 0.154 мДж/см$^2$ & 15.8 Дж/см$^2$ & $10^5$ \\
Плотность мощности & 0.154 кВт/см$^2$ & 15.8 МВт/см$^2$ & $10^5$ \\
\hline
\end{tabular}
\end{table}

\section{Вывод}

Рабочие параметры на всех оптических элементах на 3-5 порядков ниже повреждающих порогов. Система имеет значительный запас прочности и исключает риск оптического повреждения.

\endinput