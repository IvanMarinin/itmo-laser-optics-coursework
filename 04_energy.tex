\chapter{Энергетические требования к технологии маркировки}
%Оценка плотности энергии / плотности мощности, необходимой для выполнения технологической операции.
\section{Тепловая модель лазерной обработки}

Для определения энергетических требований к лазерному излучению используем аналитическую тепловую модель. Основные допущения:
\begin{itemize}
    \item Обрабатываемый материал: стекло (наиболее требовательный материал)
    \item Диаметр лазерного пятна: $d = 60$ мкм $\Rightarrow r_0 = 30$ мкм
    \item Требуемая температура: $T_{\text{разр}} = 1700^\circ\text{C}$ (разрушение стекла)
    \item Начальная температура: $T_{\text{нач}} = 20^\circ\text{C}$
    \item Коэффициент отражения: $R = 0.1$
\end{itemize}

\section{Определение режима нагрева}

Теплофизические параметры стекла:
\begin{align*}
    \rho &= 2300 \, \text{кг/м}^3 \\
    c &= 800 \, \text{Дж/кг·К} \\
    k &= 0.75 \, \text{Вт/м·К}
\end{align*}

Коэффициент температуропроводности:
\[
a = \frac{k}{\rho c} = \frac{0.75}{2300 \cdot 800} = 4.07 \cdot 10^{-7} \, \text{м}^2/\text{с}
\]

Характерный размер тепловой диффузии за время импульса $\tau = 1$ мкс:
\[
\sqrt{a\tau} = \sqrt{4.07 \cdot 10^{-7} \cdot 1 \cdot 10^{-6}} = 6.38 \cdot 10^{-7} \, \text{м} = 0.638 \, \text{мкм}
\]

Поскольку $r_0 = 30$ мкм $> \sqrt{a\tau} = 0.638$ мкм, воспользуемся формулой для импульсного нагрева.

\section{Оценка требуемой мощности}

Из методического пособия "<Лазерные технологии в задачах и примерах"> возьмем формулу для оценки температуры тела в случае импульсного нагревания~\cite{VeikoShahno}:
\[
T - T_H = \dfrac{2q_0 (1 - R) \sqrt{a \tau}}{k \sqrt{\pi}}
\]

Выразим требуемую плотность мощности из формулы импульсного нагрева:
\[
q_0 = \frac{(T - T_H) k \sqrt{\pi}}{2(1 - R) \sqrt{a \tau}}
\]

Подставим значения:
\[
q_0 = \frac{(1700 - 20) \cdot 0.75 \cdot \sqrt{3.1416}}{2(1 - 0.1) \cdot \sqrt{4.07 \cdot 10^{-7} \cdot 1 \cdot 10^{-6}}} = 3.87 \cdot 10^{9} \, \text{Вт/м}^2
\]

Требуемая импульсная мощность лазера:
\[
P = q_0  \pi r_0^2 = 3.87 \cdot 10^{9} \cdot 2.827 \cdot 10^{-9} = 10.94 \, \text{Вт}
\]

Полученные энергетические параметры обеспечивают надежную и качественную лазерную маркировку стекла с запасом по мощности.
\endinput