%Оценка плотности энергии / плотности мощности, необходимой для выполнения технологической операции.
\chapter{Энергетические требования к технологии маркировки}

\section{Параметры излучения}

Для выполнения расчетов используем следующие параметры лазерного излучения:
\begin{itemize}
    \item Длина волны: $\lambda = 10.6$ мкм
    \item Диаметр лазерного пятна: $d = 60$ мкм
    \item Радиус пятна: $r_0 = 30$ мкм
    \item Длительность импульса: $\tau = 1$ мкс
    \item Начальная температура: $T_{\text{нач}} = 20^\circ\text{C}$
\end{itemize}

\section{Теплофизические параметры материалов}
Возьмем теплофизические параметры для стекла, керамики, ПММА из методического пособия "<Лазерные технологии в задачах и примерах">~\cite{VeikoShahno}.
\subsection{Стекло}
\begin{align*}
    \rho &= 2300 \, \text{кг/м}^3 \\
    c &= 800 \, \text{Дж/кг·К} \\
    k &= 0.75 \, \text{Вт/м·К} \\
    T_p &= 1700^\circ\text{C} \\
    R &= 0.1
\end{align*}

Коэффициент температуропроводности:
\[
a = \frac{k}{\rho c} = \frac{0.75}{2300 \cdot 800} = 4.07 \cdot 10^{-7} \, \text{м}^2/\text{с}
\]

\subsection{Керамика}
\begin{align*}
    \rho &= 1500 \, \text{кг/м}^3 \\
    c &= 600 \, \text{Дж/кг·К} \\
    k &= 0.8 \, \text{Вт/м·К} \\
    T_p &= 1500^\circ\text{C} \\
    R &= 0.1
\end{align*}

Коэффициент температуропроводности:
\[
a = \frac{k}{\rho c} = \frac{0.8}{1500 \cdot 600} = 8.89 \cdot 10^{-7} \, \text{м}^2/\text{с}
\]

\subsection{ПММА}
\begin{align*}
    \rho &= 2200 \, \text{кг/м}^3 \\
    c &= 1500 \, \text{Дж/кг·К} \\
    k &= 0.5 \, \text{Вт/м·К} \\
    T_p &= 600^\circ\text{C} \\
    R &= 0.5
\end{align*}

Коэффициент температуропроводности:
\[
a = \frac{k}{\rho c} = \frac{0.5}{2200 \cdot 1500} = 1.52 \cdot 10^{-7} \, \text{м}^2/\text{с}
\]

\section{Определение режима нагрева}

Для всех материалов выполняется проверка условия импульсного нагрева:

\begin{itemize}
    \item \textbf{Стекло}: $\sqrt{a\tau} = \sqrt{4.07 \cdot 10^{-7} \cdot 1 \cdot 10^{-6}} = 0.638$ мкм < $r_0 = 30$ мкм
    \item \textbf{Керамика}: $\sqrt{a\tau} = \sqrt{8.89 \cdot 10^{-7} \cdot 1 \cdot 10^{-6}} = 0.943$ мкм < $r_0 = 30$ мкм
    \item \textbf{ПММА}: $\sqrt{a\tau} = \sqrt{1.52 \cdot 10^{-7} \cdot 1 \cdot 10^{-6}} = 0.389$ мкм < $r_0 = 30$ мкм
\end{itemize}

Во всех случаях характерный размер тепловой диффузии значительно меньше радиуса пятна, что подтверждает применимость модели импульсного нагрева.

\section{Оценка плотности мощности}

Используем формулу импульсного нагрева из методического пособия "<Лазерные технологии в задачах и примерах">~\cite{VeikoShahno}.:
\[
q_0 = \frac{(T_p - T_{\text{нач}}) k \sqrt{\pi}}{2(1 - R) \sqrt{a \tau}}
\]

\subsection{Расчет для стекла}
\[
q_0 = \frac{(1700 - 20) \cdot 0.75 \cdot \sqrt{3.1416}}{2(1 - 0.1) \cdot \sqrt{4.07 \cdot 10^{-7} \cdot 1 \cdot 10^{-6}}} = 194.3 \, \text{кВт/см}^2
\]

\subsection{Расчет для керамики}
\[
q_0 = \frac{(1500 - 20) \cdot 0.8 \cdot \sqrt{3.1416}}{2(1 - 0.1) \cdot \sqrt{8.89 \cdot 10^{-7} \cdot 1 \cdot 10^{-6}}} = 123.6 \, \text{кВт/см}^2
\]

\subsection{Расчет для пластмассы (ПММА)}
\[
q_0 = \frac{(600 - 20) \cdot 0.5 \cdot \sqrt{3.1416}}{2(1 - 0.5) \cdot \sqrt{1.52 \cdot 10^{-7} \cdot 1 \cdot 10^{-6}}} = 132.0 \, \text{кВт/см}^2
\]
\newpage
\section{Сравнительный анализ}

\begin{table}[H]
\caption{Сравнение энергетических параметров для различных материалов}
\centering
\begin{tabular}{ l c c }
\hline
Материал & $q_0$, кВт/см$^2$ & $P_{\text{треб}}$, Вт \\
\hline
Стекло & 194.3 & 5.49 \\
Керамика & 123.6 & 3.49 \\
ПММА & 132.0 & 3.73 \\
\hline
\end{tabular}
\end{table}

\section{Вывод}

Расчеты показывают, что для всех трех материалов требуемая импульсная мощность не превышает 5.5 Вт. Импульсная мощность выбранного лазера составляет порядка 400 Вт, что обеспечивает более чем 70-кратный запас по мощности для наиболее требовательного материала --- стекла. Такой значительный запас позволяет компенсировать потери в оптической системе и обеспечивает надежную работу при изменении условий обработки.

\endinput