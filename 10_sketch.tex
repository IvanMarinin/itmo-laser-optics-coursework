\chapter{Эскизный проект оптомеханической системы}
% Эскизный чертеж оптомеханической системы лазерного комплекса.

Оптомеханическая система лазерного комплекса предназначена для точного позиционирования лазерного пучка в трёхмерном пространстве и его фокусировки в зоне обработки. Выбранная компоновка системы обеспечивает гибкость управления, устойчивость и удобство эксплуатации.

Система включает в себя следующие ключевые компоненты:
\begin{itemize}
    \item Лазерный источник~\cite{Laser}.
    \item Расширитель пучка~\cite{Tele}.
    \item Четыре зеркала~\cite{Miror}.
    \item Фокусирующая линза~\cite{ALens}.
\end{itemize}

Принцип работы системы основан на разделении степеней свободы по осям координат:
\begin{itemize}
    \item Перемещение пятна обработки в плоскости XY осуществляется перемещением двух подвижных зеркал.
    \item Корректировка положения перетяжки по оси Z обеспечивается перемещением фокусирующей линзы.
\end{itemize}

\section{Функциональное назначение компонентов}

\subsection{Статические зеркала}
Первые два зеркала закреплены статично. Их основная функция --- изменение траектории пучка для организации компоновки. Опуская лазерный источник относительно рабочего поля, система обеспечивает легкий и безопасный доступ к излучателю для его обслуживания и юстировки.

\subsection{Подвижные зеркала}
Третье и четвертое зеркала подвижны. Эти зеркала отвечают за перемещение пучка и точное позиционирование фокального пятна в рабочей плоскости по осям X и Y.

\subsection{Фокусирующая линза}
Фокусирующая линза установлена в отдельном модуле, оснащенном приводом линейного перемещения вдоль оптической оси. Что позволяет перемещать перетяжку пучка по оси Z, и тем самым работать с трехмерными объектами, поддерживая минимальный размер пятна в зоне обработки.

\begin{figure}[H]
    \centering
    \includegraphics[width=1\textwidth]{Схема.png}
    \caption{Эскиз оптомеханической системы}
\end{figure}

\endinput