\chapter*{Введение}
\addcontentsline{toc}{chapter}{Введение}
Современное производство всё чаще требует высокоточной обработки крупногабаритных материалов — от стеклянных панелей в архитектуре до керамических компонентов в электронике. Для решения многих из этих задач оптимально подходят лазерные плоттеры — портальные системы, в которых оптика перемещается над неподвижной заготовкой. Такой подход позволяет обрабатывать детали большого размера без потери качества по краям поля~\cite{bann2010laser}.

Для обработки органических и силикатных материалов часто используют именно CO$_\text{2}$ лазер. Стекло, оргстекло (ПММА) и керамика интенсивно поглощают излучение на длине волны 10.6 мкм, что обеспечивает чистую и контролируемую абляцию поверхности~\cite{mckee2021laser}. Импульсные CO$_\text{2}$ лазеры способны создавать структуры с разрешением до 50 мкм — именно то, что нужно для прецизионной маркировки~\cite{temmler2021investigation}.

Главное преимущество такой системы — стабильность. В отличие от гальванометрических сканаторов, где луч отклоняется зеркалами, в портальной системе угол падения излучения на материал остаётся постоянным. Это значит, что качество гравировки в центре и по краям рабочего поля будет одинаковым.

Области применения портальных лазерных граверов:
\begin{itemize}
    \item Архитектура и дизайн: декоративная гравировка на стеклянных перегородках, керамической плитке;
    \item Промышленность: маркировка серийных номеров на технической керамике, структуризация поверхности;
    \item Рекламная продукция: создание сувениров из акрила, указателей, логотипов.
\end{itemize}

Система автофокусировки по оси Z позволяет работать с материалами разной толщины и рельефа, автоматически поддерживая оптимальное положение фокальной плоскости. Это делает комплекс универсальным инструментом для современных производственных задач.
\endinput