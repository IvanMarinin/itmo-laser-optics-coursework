% Определение характеристик лазерного пучка в исходной перетяжке: радиус перетяжки, расстояние от перетяжки до выходного окна лазера, длина Рэлея.
\chapter{Параметры лазерного пучка в исходной перетяжке}

\section{Исходные параметры лазерного пучка}

Для расчета характеристик пучка в исходной перетяжке используем параметры выбранного лазера BC-10:
\begin{itemize}
    \item Длина волны: $\lambda = 10.6$ мкм
    \item Диаметр пучка на выходном окне: $D_w = 3.5$ мм
    \item Расходимость пучка: $2\theta = 8$ мрад
    \item Качество пучка: $M^2 = 1.1$
\end{itemize}

\section{Расчет радиуса перетяжки пучка}

\[
\omega_{01} = \frac{M^2 \lambda}{\pi \theta} = \frac{1.1 \cdot 10.6 \cdot 10^{-6}}{\pi \cdot 4 \cdot 10^{-3}} = 0.928 \, \text{мм}
\]

\section{Расчет длины Рэлея}

\[
z_{R1} = \frac{\pi \omega_{01}^2}{M^2 \lambda} = \frac{\pi \cdot (0.928 \cdot 10^{-3})^2}{1.1 \cdot 10.6 \cdot 10^{-6}} = 0.23 \, \text{м}
\]

\section{Расчет положения перетяжки относительно выходного окна}

\begin{align*}
z_0 &= z_{R1} \sqrt{\left(\frac{D_w \pi \theta}{2 M^2 \lambda}\right)^2 - 1} \\
z_0 &= 0.23 \sqrt{\left(\frac{3.5 \cdot 10^{-3} \cdot \pi \cdot 4 \cdot 10^{-3}}{2 \cdot 1.1 \cdot 10.6 \cdot 10^{-6}}\right)^2 - 1} = 0.063 \, \text{м}
\end{align*}

\endinput