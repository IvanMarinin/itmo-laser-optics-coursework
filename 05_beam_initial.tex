% Определение характеристик лазерного пучка в исходной перетяжке: радиус перетяжки, расстояние от перетяжки до выходного окна лазера, длина Рэлея.
\chapter{Параметры лазерного пучка в исходной перетяжке}

\section{Исходные параметры лазерного пучка}

Для расчета характеристик пучка в исходной перетяжке используем параметры лазера:
\begin{itemize}
    \item Длина волны: $\lambda = 10600$ нм = $10.6$ мкм
    \item Диаметр пучка: $D_w = 3.6$ мм
    \item Полный угол расходимости: $2\theta = 5$ мрад
    \item Полуугол расходимости: $\theta = 2.5$ мрад
    \item Качество пучка: $M^2 = 1.3$
\end{itemize}

\section{Расчет радиуса перетяжки пучка}

\[
\omega_{01}
= \dfrac{M^2\,\lambda}{\pi\,\theta}
= \dfrac{1.3 \cdot 10.6 \cdot 10^{-6}}{\pi \cdot 2.5 \cdot 10^{-3}} 
\approx 1.754 \, \text{мм}
\]

\section{Расчет длины Рэлея}

\[
z_{R1}
= \dfrac{M^2\,\lambda}{\pi\,\theta^2}
= \dfrac{1.3 \cdot 10.6 \cdot 10^{-6}}{\pi \cdot (2.5 \cdot 10^{-3})^2}
\approx 702 \, \text{мм}
\]

\section{Расчет положения перетяжки относительно выходного окна}

\begin{align*}
   z_0
    &= z_{R1}\,\sqrt{\left(\dfrac{D_w\cdot\pi\cdot\theta}{2\cdot M^2\cdot\lambda}\right)^2 - 1}\\
    &= 0.702\,\sqrt{\left(\dfrac{3.6\cdot 10^{-3}\cdot\pi\cdot 2.5\cdot 10^{-3}}{2\cdot 1.3\cdot 10.6\cdot 10^{-6}}\right)^2 - 1} \approx 161 \, \text{мм} 
\end{align*}

\endinput