\chapter{Заключение}
%Выводы
В ходе выполнения курсового проекта была успешно спроектирована портальная лазерная система для маркировки крупногабаритных материалов. Основным результатом работы стало создание законченной оптико-механической схемы, удовлетворяющей всем техническим требованиям.

Ключевым достижением проекта стало решение проблемы нестабильности диаметра пучка в оптической системе. Исходные расчеты показали, что без коррекции диаметр пучка на фокусирующей линзе изменялся бы от 5,17 до 44,67 мм, что делало невозможным использование единой фокусирующей оптики. Эта проблема была успешно решена введением в схему телескопического расширителя пучка 5$^×$, который стабилизировал диаметр в диапазоне 17,55-19,60 мм.

Важным этапом стал выбор фокусирующей оптики. Анализ показал, что стандартная сферическая линза приводит к недопустимому уширению пятна до 109,3 мкм из-за сферических аберраций. В качестве решения была выбрана асферическая линза с фокусным расстоянием 50,8 мм, обеспечивающая формирование пятна диаметром 45,5-50,7 мкм по всему рабочему полю.

Энергетические расчеты подтвердили достаточность мощности выбранного лазера 40 Вт для обработки целевых материалов. Тепловая модель показала требуемую мощность 10,94 Вт, что обеспечивает четырехкратный запас. Расчеты плотностей энергии на оптических элементах показали значительный запас прочности - рабочие параметры на 3-5 порядков ниже повреждающих порогов.

Динамические расчеты определили параметры сканирования: шаг 12 мкм при перекрытии 80\%, минимальная скорость 1,2 м/с, шаг коррекции фокуса 48,4 мкм. Установлена реалистичная длительность разгона системы - 40 мс.

Разработанная оптомеханическая схема обеспечивает точное позиционирование лазерного пятна в трехмерном пространстве при сохранении эргономичного доступа к оборудованию. Все компоненты системы подобраны из коммерчески доступных изделий, что подтверждает практическую реализуемость проекта.

Таким образом, проектом доказана возможность создания портальной лазерной системы для высокоточной маркировки крупногабаритных изделий с соблюдением всех технических требований по качеству пятна, мощности и надежности.
\endinput