%Выбор источника лазерного излучения, параметры которого позволят выполнить поставленную задачу. Определение параметров лазерного излучения: длина волны, качество пучка (М2), профиль интенсивности, расходимость, диаметр пучка на выходном окне, средняя мощность / энергия в импульсе, длительность и частота следования импульсов.
\chapter{Выбор и параметры лазерного источника}

Для выполнения задачи лазерной маркировки стекла, ПММА и керамики на поле 5×3 метра был выбран CO$_2$ лазер. Данный выбор обоснован эффективным поглощением излучения с длиной волны 10.6 мкм всеми перечисленными материалами.

\section{Выбранная модель лазера}

В качестве лазерного источника выбрана модель \textbf{BC-10} производства Nanjing Wavelength Opto-Electronic Science and Technology Co.

\begin{figure}[H]
    \centering
    \includegraphics[width=0.75\textwidth]{Laser.jpg}
    \caption{Лазерный источник BC-10}
\end{figure}

Со следующими параметрами~\cite{Laser}:

\begin{table}[H]
\caption{Параметры лазерного излучения}
\centering
\begin{tabular}{lc}
\hline
\textbf{Параметр} & \textbf{Значение} \\
\hline
Длина волны & 10.57-10.63 мкм \\
Выходная мощность & $\geq$12 Вт \\
Стабильность мощности & $\pm$10\% \\
Модовая структура & TEM$_{00}$ \\
Диаметр пучка на выходе & 3.5 мм \\
Расходимость пучка & 4 мрад \\
Длительность импульса & 150 мкс \\
Частота модуляции & 0-20 кГц \\
Поляризация & Линейная, 50:1 \\
Охлаждение & Воздушное \\
Срок службы & 20000 часов \\
\hline
\end{tabular}
\end{table}
\section{Обоснование выбора}

Длина волны 10.6 мкм обеспечивает эффективное поглощение всеми целевыми материалами: стеклом (колебания решетки SiO$_2$), ПММА (колебания связей C-O и C=O) и керамикой (колебания кристаллической решетки).

\subsection{Расчет диаметра фокусного пятна}

Качество пучка TEM$_{00}$ с расходимостью 4 мрад гарантирует формирование малого фокусного пятна. Оценим диаметр пятна в фокусе линзы:

\[
d = \frac{4\lambda f}{\pi D}
\]

где:
\begin{itemize}
    \item $\lambda = 10.6$ мкм - длина волны излучения
    \item $f = 10$ мм - фокусное расстояние линзы
    \item $D = 3.5$ мм - диаметр пучка на линзе
\end{itemize}

Подставляя значения, получаем:
\[
d = \frac{4 \cdot 10.6 \cdot 10^{-6} \cdot 0.01}{\pi \cdot 0.0035} \approx 39 \, \text{мкм}
\]

Таким образом, при использовании линзы с фокусным расстоянием 10 мм возможно формирование пятна диаметром приблизительно 39 мкм.



\subsection{Оценка достаточности мощности}

Мощности 12 Вт достаточно для обработки указанных материалов при оптимальных скоростях. Плотность мощности в фокальном пятне:

Площадь пятна:
\[
S = \frac{\pi d^2}{4} = \frac{\pi (39 \cdot 10^{-6})^2}{4} \approx 1.19 \cdot 10^{-9} \, \text{м}^2
\]

Импульсная плотность мощности:
\[
q_{\text{имп}} = \frac{P}{f \cdot \tau \cdot S} = \frac{12}{20000 \cdot 150 \cdot 10^{-6} \cdot 1.19 \cdot 10^{-9}} \approx 3.36 \, \text{кВт/мм}^2
\]

Данной плотности мощности достаточно для эффективной абляции стекла, ПММА и керамики.

Воздушное охлаждение упрощает конструкцию системы, а наработка на отказ 20000 часов обеспечивает долговременную надежность в промышленной эксплуатации.

Выбранный лазерный источник полностью соответствует требованиям технологической задачи и обеспечивает необходимые характеристики для маркировки указанных материалов на крупногабаритном поле.

\endinput