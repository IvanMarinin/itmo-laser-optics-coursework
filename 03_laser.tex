%Выбор источника лазерного излучения, параметры которого позволят выполнить поставленную задачу. Определение параметров лазерного излучения: длина волны, качество пучка (М2), профиль интенсивности, расходимость, диаметр пучка на выходном окне, средняя мощность / энергия в импульсе, длительность и частота следования импульсов.
\chapter{Выбор и параметры лазерного источника}

Для выполнения задачи лазерной маркировки стекла, ПММА и керамики на поле 5×3 метра был выбран CO$_2$ лазер.

\section{Выбранная модель лазера}

В качестве лазерного источника выбрана модель \textbf{Coherent Diamond Series 1322624} производства Coherent Inc.~\cite{Laser}.

\begin{figure}[H]
    \centering
    \includegraphics[width=0.75\textwidth]{Laser.jpg}
    \caption{Лазерный источник Coherent Diamond Series}
\end{figure}
\newpage
Со следующими параметрами:

\begin{table}[H]
\caption{Параметры лазерного излучения}
\centering
\begin{tabular}{lc}
\hline
\textbf{Параметр} & \textbf{Значение} \\
\hline
Длина волны & 10.6 мкм \\
Выходная мощность & 40 Вт \\
Стабильность мощности & $\pm$5\% \\
Модовая структура & TEM$_{00}$ \\
Диаметр пучка на выходе & 1.8 мм \\
Расходимость пучка & <5 мрад \\
Длительность импульса & 1 мкс \\
Частота модуляции & 0-100 кГц \\
Поляризация & Линейная, >100:1 \\
Охлаждение & Воздушное \\
Качество пучка M² & <1.3 \\
\hline
\end{tabular}
\end{table}

\section{Обоснование выбора}

Излучение с длиной волны 10.6 мкм эффективно поглощается такими материалами, как стекло (колебания решетки SiO$_2$), ПММА (колебания связей C-O и C=O) и керамика (колебания кристаллической решетки).

\subsection{Оценка достаточности мощности}

Мощности 40 Вт достаточно для обработки указанных материалов. Плотность мощности в фокальном пятне:

Площадь пятна:
\[
S = \frac{\pi d^2}{4} = \frac{\pi (60 \cdot 10^{-6})^2}{4} \approx 2.82 \cdot 10^{-9} \, \text{м}^2
\]

Импульсная плотность мощности:
\[
q_{\text{имп}} = \frac{P}{f \cdot \tau \cdot S} = \frac{40}{100000 \cdot 1 \cdot 10^{-6} \cdot 2.82 \cdot 10^{-9}} \approx 14.18 \, \text{МВт/см}^2
\]

Данной плотности мощности достаточно для эффективной абляции стекла, ПММА и керамики.

\subsection{Длительность импульса}
Для лазерной маркировки стеклянных материалов применяются лазерные источники с различной длительностью импульсов, охватывающей широкий диапазон от наносекунд до миллисекунд. В исследованиях демонстрируется успешное использование как сверхкоротких импульсов с длительностью пика 162~нс и хвоста 29.6~мкс~\cite{ShortImp}, так и более длинных импульсов в диапазоне от 10~мкс до 1000~мкс~\cite{LongImp}. Выбранная для данного проекта длительность импульса 1~мкс занимает промежуточное положение в этом диапазоне и обеспечивает баланс между эффективностью абляции и минимальным тепловым воздействием на материал, что позволяет значительно снизить риск возникновения трещин при маркировке.

\subsection{Конструктивные параметры}
Конструктивные особенности выбранного лазерного источника обеспечивают значительные преимущества при интеграции в портальную систему. Воздушное охлаждение исключает необходимость использования дорогостоящих жидкостных систем теплоотвода, что существенно упрощает общую конструкцию установки, повышает её надежность и снижает эксплуатационные расходы.

\subsection{Качество пучка}
Высокое качество пучка с параметром M² < 1.3 гарантирует возможность формирования минимального фокального пятна, что является критически важным для достижения высокого разрешения маркировки и четкости наносимых изображений.

\subsection{Заключение}
Выбранный лазерный источник полностью соответствует требованиям технологической задачи и обеспечивает необходимые характеристики для маркировки указанных материалов на крупногабаритном поле.

\endinput