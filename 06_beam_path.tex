% Расчет пространственных характеристик лазерного пучка от выходного окна до плоскости обработки. Описать ход лазерного пучка в оптической системе.
\chapter{Распространение пучка в оптической системе}

\section{Описание оптической системы}

Оптическая система доставки лазерного излучения к зоне обработки включает четыре зеркала и фокусирующую линзу. Зеркала обеспечивают отклонение пучка по осям X, Y и Z, а фокусирующая линза формирует пятно в плоскости обработки. 

\section{Расчёт расстояний до фокусирующей линзы}

Геометрия портальной системы определяет изменение расстояния от исходной перетяжки пучка до фокусирующей линзы. Учитываем следующие составляющие:
\begin{itemize}
    \item Расстояние от перетяжки до выходного окна лазера: 0.368 м
    \item Расстояние от выходного окна до первого зеркала: 0.2 м
    \item Расстояние между первым и вторым зеркалами: 0.4 м
    \item Перемещение по оси X: от 0 до 5 м
    \item Перемещение по оси Y: от 0 до 3 м
    \item Перемещение по оси Z: от 0 до 0.15 м
\end{itemize}

Минимальное расстояние:
\[
s_{\text{min}} = 0.368 + 0.2 + 0.4 + 0 + 0 + 0 = 0.968 \, \text{м}
\]

Центральное положение:
\[
s_{\text{center}} = 0.368 + 0.2 + 0.4 + 2.5 + 1.5 + 0.075 = 5.043 \, \text{м}
\]


Максимальное расстояние:
\[
s_{\text{max}} = 0.368 + 0.2 + 0.4 + 5 + 3 + 0.15 = 9.118 \, \text{м}
\]


\section{Расчёт диаметра пучка на фокусирующей линзе}

Диаметр пучка на фокусирующей линзе рассчитывается по формуле:
\[
D = 2\omega_{01} \sqrt{1 + \left(\frac{s - z_0}{z_{R1}}\right)^2}
\]

\begin{itemize}
    \item Для $s_{\text{min}} = 0.968$ м: $D_{\text{min}} = 5.18$ мм
    \item Для $s_{\text{center}} = 5.043$ м: $D_{\text{center}} = 37.77$ мм
    \item Для $s_{\text{max}} = 9.118$ м: $D_{\text{max}} = 70.62$ мм
\end{itemize}

\section{Вывод формулы для фокусного расстояния}

Требуемый диаметр пятна в фокусе: $d = 60$ мкм, что соответствует радиусу перетяжки $\omega_{02} = 30$ мкм. 
Увеличение линзы определяется как:
\[
\Gamma = \frac{\omega_{02}}{\omega_{01}} = \frac{0.030}{0.928} \approx 0.0323
\]

Найдем фокусное расстояние формулы преобразования пучка линзой:
\[
\Gamma = \frac{f}{\sqrt{(s - f)^2 + z_{R1}^2}}
\]

Решаем уравнение относительно $f$. После возведения в квадрат и преобразований получаем квадратное уравнение:
\[
(\Gamma^2 - 1)f^2 - 2\Gamma^2 s f + \Gamma^2(s^2 + z_{R1}^2) = 0
\]

Решение уравнения:
\[
f = \frac{\Gamma^2 s - \Gamma \sqrt{s^2 + (1 - \Gamma^2)z_{R1}^2}}{\Gamma^2 - 1}
\]

\section{Расчёт фокусного расстояния для различных положений}

Для минимального расстояния $s_{\text{min}} = 0.968$ м
\[
f_{\text{min}} \approx 31.15 \, \text{мм}
\]

Для центрального положения $s_{\text{center}} = 5.043$ м
\[
f_{\text{center}} \approx 158.0 \, \text{мм}
\]

Для максимального расстояния $s_{\text{max}} = 9.118$ м
\[
f_{\text{max}} \approx 285.3 \, \text{мм}
\]

\section{Анализ результатов}

\begin{itemize}
    \item Диаметр пучка на фокусирующей линзе изменяется в 13.6 раза (от 5.18 мм до 70.62 мм)
    \item Требуемое фокусное расстояние изменяется от 31.15 мм до 285.3 мм (в 9.2 раз)
    \item Невозможно обеспечить стабильный размер фокального пятна одной фокусирующей линзой
\end{itemize}

\section{Вывод о необходимости телескопической системы}

Значительное изменение диаметра пучка на фокусирующей линзе и соответствующее изменение требуемого фокусного расстояния демонстрирует необходимость использования телескопической системы для стабилизации диаметра пучка.







\section{Применение телескопической системы GBE10-E3}

Для стабилизации диаметра пучка в оптической системе используется коммерчески доступный телескоп GBE10-E3 производства Thorlabs с увеличением 10x, устанавливаемый непосредственно на выходное окно лазера~\cite{Tele}.


\begin{figure}[H]
    \centering
    \includegraphics[width=0.75\textwidth]{Tele.jpg}
    \caption{Расширитель пучка GBE10-E3}
\end{figure}

\subsection{Параметры телескопа GBE10-E3}

\begin{itemize}
    \item Увеличение: $\beta = 10$
    \item Максимальный входной диаметр пучка: 3.5 мм
    \item Просветляющее покрытие: для 10.6 мкм
    \item Типичное пропускание: >94\%
\end{itemize}

\subsection{Параметры пучка после телескопа}

После прохождения телескопа параметры пучка составляют:

\begin{align*}
    \omega_{0,\text{нов}} &= \beta \cdot \omega_{01} = 10 \cdot 0.928 = 9.28 \, \text{мм} \\
    z_{R,\text{нов}} &= \beta^2 \cdot z_{R1} = 100 \cdot 0.23 = 23 \, \text{м} \\
    \theta_{\text{нов}} &= \frac{\theta}{\beta} = \frac{4}{10} = 0.4 \, \text{мрад}
\end{align*}

\subsection{Расстояния до фокусирующей линзы после телескопа}

Поскольку телескоп установлен на выходном окне лазера, расстояние от новой перетяжки до фокусирующей линзы:

\begin{align*}
    s'_{\text{min}} &= 0.2 + 0.4 + 0 + 0 + 0 = 0.6 \, \text{м} \\
    s'_{\text{center}} &= 0.2 + 0.4 + 2.5 + 1.5 + 0.075 = 4.675 \, \text{м} \\
    s'_{\text{max}} &= 0.2 + 0.4 + 5 + 3 + 0.15 = 8.75 \, \text{м}
\end{align*}

\subsection{Стабилизация диаметра пучка}

Диаметр пучка на фокусирующей линзе после телескопа:

\begin{align*}
    D_{\text{min}} &= 2\omega_{0,\text{нов}} \sqrt{1 + \left(\frac{s'_{\text{min}}}{z_{R,\text{нов}}}\right)^2} = 18.57 \, \text{мм} \\
    D_{\text{center}} &= 2\omega_{0,\text{нов}} \sqrt{1 + \left(\frac{s'_{\text{center}}}{z_{R,\text{нов}}}\right)^2} = 18.94 \, \text{мм} \\
    D_{\text{max}} &= 2\omega_{0,\text{нов}} \sqrt{1 + \left(\frac{s'_{\text{max}}}{z_{R,\text{нов}}}\right)^2} = 19.85 \, \text{мм}
\end{align*}

Относительное изменение диаметра: $\frac{19.85}{18.57} = 1.069$ раза (6.9\%)

\subsection{Требуемое фокусное расстояние}



Диаметр пятна в плоскости обработки составляет 56.1-60.0 мкм, что удовлетворяет требованиям технического задания (50-70 мкм).

\section{Вывод}

Применение телескопа GBE10-E3 с увеличением 10x позволяет достичь значительной стабилизации параметров пучка:
\begin{itemize}
    \item Диаметр пучка на фокусирующей линзе изменяется всего на 6.9\% (против 1360\% без телескопа)
    \item Требуемое фокусное расстояние практически постоянно (разброс 0.9 мм)
    \item Размер фокального пятна стабилен и находится в требуемом диапазоне 50-70 мкм
    \item Система использует коммерчески доступные компоненты и проста в юстировке
\end{itemize}


\subsection{Расчет фокусного расстояния и параметров фокального пятна}

Для точного расчета фокусного расстояния и параметров фокального пятна используем формулы преобразования гауссова пучка одиночной линзой.

Увеличение линзы после телескопа:
\[
\Gamma = \frac{\omega_{02}}{\omega_{0,\text{нов}}} = \frac{0.030}{9.28} \approx 0.00323
\]

Требуемое фокусное расстояние рассчитывается по формуле:
\[
f = \frac{\Gamma^2 s - \Gamma \sqrt{s^2 + (1 - \Gamma^2)z_{R,\text{нов}}^2}}{\Gamma^2 - 1}
\]

Расчет для различных положений:

\begin{itemize}
    \item Для $s'_{\text{min}} = 0.6$ м:\\ 
        $f_{\text{min}} = \dfrac{0.00323^2 \cdot 0.6 - 0.00323 \cdot \sqrt{0.6^2 + (1 - 0.00323^2) \cdot 23^2}}{0.00323^2 - 1}$\\ $ \approx 74.4 \, \text{мм}$
    
    \item Для $s'_{\text{center}} = 4.675$ м:\\ 
        $f_{\text{center}} = \dfrac{0.00323^2 \cdot 4.675 - 0.00323 \cdot \sqrt{4.675^2 + (1 - 0.00323^2) \cdot 23^2}}{0.00323^2 - 1} $\\ $ \approx 75.0 \, \text{мм}$
    
    \item Для $s'_{\text{max}} = 8.75$ м:\\ 
        $f_{\text{max}} = \dfrac{0.00323^2 \cdot 8.75 - 0.00323 \cdot \sqrt{8.75^2 + (1 - 0.00323^2) \cdot 23^2}}{0.00323^2 - 1} $\\ $ \approx 75.6 \, \text{мм}$
\end{itemize}

Разброс фокусных расстояний: 1.2 мм

\subsection{Фокусирующая линза}

Для формирования пятна диаметром 60 мкм выбрана линза с фокусным расстоянием $f = 75$ мм и диаметром 25.4 мм. 

Расчет радиуса перетяжки после линзы выполняется по формуле:
\[
\Gamma = \frac{f}{\sqrt{(s - f)^2 + z_{R,\text{нов}}^2}}, \quad \omega_{02} = \Gamma \cdot \omega_{0,\text{нов}}
\]

Результаты расчета диаметра фокального пятна ($d = 2\omega_{02}$):

\begin{align*}
    \Gamma_{\text{min}} &= \frac{0.075}{\sqrt{(0.6 - 0.075)^2 + 23^2}} \approx 0.003260,\\ d_{\text{min}} &= 2 \cdot 0.003260 \cdot 9.28 \approx 60.5 \, \text{мкм} \\
    \Gamma_{\text{center}} &= \frac{0.075}{\sqrt{(4.675 - 0.075)^2 + 23^2}} \approx 0.003197,\\ d_{\text{center}} &= 2 \cdot 0.003197 \cdot 9.28 \approx 59.3 \, \text{мкм} \\
    \Gamma_{\text{max}} &= \frac{0.075}{\sqrt{(8.75 - 0.075)^2 + 23^2}} \approx 0.003051,\\ d_{\text{max}} &= 2 \cdot 0.003051 \cdot 9.28 \approx 56.6 \, \text{мкм}
\end{align*}

Диаметр пятна в плоскости обработки составляет 56.6-60.5 мкм, что удовлетворяет требованиям технического задания (50-70 мкм).


\subsection{Расчет сферических аберраций для двояковыпуклой линзы B7834-E3}

Для оценки влияния сферических аберраций на качество фокального пятна проведем расчет коэффициентов сферических аберраций для выбранной двояковыпуклой линзы B7834-E3~\cite{LensBi}.

\subsubsection{Параметры линзы и системы}

\begin{itemize}
    \item Модель линзы: B7834-E3 (Thorlabs)
    \item Фокусное расстояние: $f = 75$ мм
    \item Радиусы кривизны: $R_1 = 209.6$ мм, $R_2 = -209.6$ мм
    \item Диаметр: 25.4 мм
    \item Материал: Laser-Grade Zinc Selenide (ZnSe)
    \item Показатель преломления: $n = 2.4$ при $\lambda = 10.6$ мкм
    \item Максимальный диаметр пучка на линзе: $D_L = 19.85$ мм
    \item Расстояние до линзы: $s = 8.75$ м (наихудший случай)
\end{itemize}

\subsubsection{Расчет коэффициентов сферической аберрации}

Коэффициенты формы линзы:
\begin{align*}
    a_k &= -\frac{R_2 + R_1}{R_2 - R_1} = -\frac{-209.6 + 209.6}{-209.6 - 209.6} = 0 \\
    b_k &= 1 - \frac{2f}{s} = 1 - \frac{2 \cdot 0.075}{8.75} \approx 0.9829
\end{align*}

Коэффициент сферической аберрации $K^*$:
\begin{align*}
    K^* &= \frac{1}{4n(n-1)} \left[ \frac{n+2}{n-1}a_k^2 + 4(n+1)a_k b_k + (3n+2)(n-1)b_k^2 + \frac{n^3}{n-1} \right]\\
    &= \frac{1}{4 \cdot 2.4 \cdot 1.4} \left[ \frac{4.4}{1.4} \cdot 0 + 4 \cdot 3.4 \cdot 0 \cdot 0.9829 + 9.2 \cdot 1.4 \cdot (0.9829)^2 + \frac{13.824}{1.4} \right] \\
    &\approx 1.660
\end{align*}

Уширение пятна за счет сферических аберраций:
\[
d_{ab} = \frac{K^* D_L^3}{32 f^2} = \frac{1.660 \cdot (0.01985)^3}{32 \cdot (0.075)^2} \approx 72.2 \, \text{мкм}
\]

\subsubsection{Суммарный диаметр фокального пятна}

\begin{itemize}
    \item Дифракционный размер пятна: $d_{diff} = 56.6$ мкм
    \item Уширение от аберраций: $d_{ab} = 72.2$ мкм
    \item Суммарный диаметр пятна: $d_{total} = d_{diff} + d_{ab} = 128.8$ мкм
\end{itemize}

\subsubsection{Анализ результатов}

\begin{itemize}
    \item Требуемый диаметр пятна: 50-70 мкм
    \item Фактический диаметр с учетом аберраций: 128.8 мкм
    \item Превышение: 84\% относительно верхней границы требований
\end{itemize}

Сферические аберрации двояковыпуклой линзы B7834-E3 приводят к значительному уширению фокального пятна, что делает ее непригодной для выполнения требований технического задания. Так как у двояковыпуклой линзой сферические аберрации минимальны,то необходимо использовать асферические линзы.

\subsection{Асферическая линза}
Возьмем асферическую линзу производства Edmund Оptics TECHSPEC~\cite{ALens}
\begin{itemize}
    \item Материал: Германий
    \item Диаметр: 25 мм
    \item Фокусное расстояние: 75 мм
    \item Просветляющее покрытие: для диапазона 8-12 мкм
\end{itemize}

\subsubsection{Ожидаемые характеристики фокального пятна}

С использованием асферической линзы сферические аберрации пренебрежимо малы, и диаметр фокального пятна определяется только дифракционными эффектами:

\begin{align*}
    d_{\text{min}} &= 60.5 \, \text{мкм} \\
    d_{\text{center}} &= 59.3 \, \text{мкм} \\
    d_{\text{max}} &= 56.6 \, \text{мкм}
\end{align*}

Диаметр пятна в плоскости обработки составляет 56.6-60.5 мкм, что полностью удовлетворяет требованиям технического задания (50-70 мкм).

\subsection{Заключение по оптической системе}

Комбинация телескопа и асферической линзы обеспечивает:
\begin{itemize}
    \item Стабильный диаметр пучка на фокусирующей линзе (изменение 6.9\%)
    \item Высокое качество фокального пятна без сферических аберраций
    \item Соответствие всем требованиям технического задания по размеру пятна
    \item Использование коммерчески доступных и надежных компонентов
\end{itemize}

Данная оптическая система гарантирует качественную лазерную обработку по всему рабочему полю 5×3 метра.




\endinput