% Расчет пространственных характеристик лазерного пучка от выходного окна до плоскости обработки. Описать ход лазерного пучка в оптической системе.
\chapter{Распространение пучка в оптической системе}

\section{Описание оптической системы}

Оптическая система доставки лазерного излучения к зоне обработки включает четыре зеркала и фокусирующую линзу. Зеркала обеспечивают отклонение пучка по осям X, Y и Z, а фокусирующая линза формирует пятно в плоскости обработки. 

\section{Расчёт расстояний до фокусирующей линзы}

Геометрия портальной системы определяет изменение расстояния от исходной перетяжки пучка до фокусирующей линзы. Учитываем следующие составляющие:
\begin{itemize}
    \item Расстояние от перетяжки до выходного окна лазера: 0.063 м
    \item Расстояние от выходного окна до первого зеркала: 0.2 м
    \item Расстояние между первым и вторым зеркалами: 0.4 м
    \item Перемещение по оси X: от 0 до 5 м
    \item Перемещение по оси Y: от 0 до 3 м
    \item Перемещение по оси Z: от 0 до 0.15 м
\end{itemize}

Минимальное расстояние:
\[
s_{\text{min}} = 0.063 + 0.2 + 0.4 + 0 + 0 + 0 = 0.663 \, \text{м}
\]

Центральное положение:
\[
s_{\text{center}} = 0.063 + 0.2 + 0.4 + 2.5 + 1.5 + 0.075 = 4.738 \, \text{м}
\]


Максимальное расстояние:
\[
s_{\text{max}} = 0.063 + 0.2 + 0.4 + 5 + 3 + 0.15 = 8.813 \, \text{м}
\]


\section{Расчёт диаметра пучка на фокусирующей линзе}

Диаметр пучка на фокусирующей линзе рассчитывается по формуле:
\[
D = 2\omega_{01} \sqrt{1 + \left(\frac{s - z_0}{z_{R1}}\right)^2}
\]

\begin{itemize}
    \item Для $s_{\text{min}} = 0.663$ м: $D_{\text{min}} = 7.10$ мм
    \item Для $s_{\text{center}} = 4.738$ м: $D_{\text{center}} = 46.28$ мм
    \item Для $s_{\text{max}} = 8.813$ м: $D_{\text{max}} = 72.66$ мм
\end{itemize}

\section{Вывод формулы для фокусного расстояния}

Требуемый диаметр пятна в фокусе: $d = 60$ мкм, что соответствует радиусу перетяжки $\omega_{02} = 30$ мкм. 
Увеличение линзы определяется как:
\[
\Gamma = \frac{\omega_{02}}{\omega_{01}} = \frac{0.030}{0.928} \approx 0.0323
\]

Найдем фокусное расстояние формулы преобразования пучка линзой:
\[
\Gamma = \frac{f}{\sqrt{(s - f)^2 + z_{R1}^2}}
\]

Решаем уравнение относительно $f$. После возведения в квадрат и преобразований получаем квадратное уравнение:
\[
(\Gamma^2 - 1)f^2 - 2\Gamma^2 s f + \Gamma^2(s^2 + z_{R1}^2) = 0
\]

Решение уравнения:
\[
f = \frac{\Gamma^2 s - \Gamma \sqrt{s^2 + (1 - \Gamma^2)z_{R1}^2}}{\Gamma^2 - 1}
\]

\section{Расчёт фокусного расстояния для различных положений}

Для минимального расстояния $s_{\text{min}} = 0.663$ м
\[
f_{\text{min}} \approx 22.0 \, \text{мм}
\]

Для центрального положения $s_{\text{center}} = 4.738$ м
\[
f_{\text{center}} \approx 148.4 \, \text{мм}
\]

Для максимального расстояния $s_{\text{max}} = 8.813$ м
\[
f_{\text{max}} \approx 275.8 \, \text{мм}
\]

\section{Анализ результатов}

\begin{itemize}
    \item Диаметр пучка на фокусирующей линзе изменяется в 10.2 раза (от 7.10 мм до 72.66 мм)
    \item Требуемое фокусное расстояние изменяется от 22.0 мм до 275.8 мм (в 12.5 раз)
    \item Невозможно обеспечить стабильный размер фокального пятна одной фокусирующей линзой
\end{itemize}

\section{Вывод о необходимости телескопической системы}

Значительное изменение диаметра пучка на фокусирующей линзе и соответствующее изменение требуемого фокусного расстояния демонстрирует необходимость использования телескопической системы для стабилизации диаметра пучка.
\endinput