% Расчет пространственных характеристик лазерного пучка от выходного окна до плоскости обработки. Описать ход лазерного пучка в оптической системе.
\chapter{Распространение пучка в оптической системе}

\section{Описание оптической системы}

Оптическая система доставки лазерного излучения к зоне обработки включает четыре зеркала и фокусирующую линзу. Зеркала обеспечивают отклонение пучка по осям X, Y и Z, а фокусирующая линза формирует пятно в плоскости обработки. 

\section{Расчёт расстояний до фокусирующей линзы}

Геометрия портальной системы определяет изменение расстояния от исходной перетяжки пучка до фокусирующей линзы. Учитываем следующие составляющие:
\begin{itemize}
    \item Расстояние от перетяжки до выходного окна лазера: 0.161 м
    \item Расстояние от выходного окна до первого зеркала: 0.2 м
    \item Расстояние между первым и вторым зеркалами: 0.4 м
    \item Перемещение по оси X: от 0 до 5 м
    \item Перемещение по оси Y: от 0 до 3 м
    \item Перемещение по оси Z: от 0 до 0.15 м
\end{itemize}

Минимальное расстояние:
\[
s_{\text{min}} = 0.161 + 0.2 + 0.4 + 0 + 0 + 0 = 0.761 \, \text{м}
\]

Центральное положение:
\[
s_{\text{center}} = 0.161 + 0.2 + 0.4 + 2.5 + 1.5 + 0.075 = 4.836 \, \text{м}
\]

Максимальное расстояние:
\[
s_{\text{max}} = 0.161 + 0.2 + 0.4 + 5 + 3 + 0.15 = 8.911 \, \text{м}
\]

\section{Расчёт диаметра пучка на фокусирующей линзе}

Диаметр пучка на фокусирующей линзе рассчитывается по формуле:
\[
D = 2\omega_{01} \sqrt{1 + \left(\frac{s}{z_{R1}}\right)^2}
\]

\begin{itemize}
    \item Для $s_{\text{min}} = 0.761$ м: $D_{\text{min}} = 2 \cdot 1.754 \cdot \sqrt{1 + \left(\frac{0.761}{0.702}\right)^2} \approx 5.17$ мм
    \item Для $s_{\text{center}} = 4.836$ м: $D_{\text{center}} = 2 \cdot 1.754 \cdot \sqrt{1 + \left(\frac{4.836}{0.702}\right)^2} \approx 24.42$ мм
    \item Для $s_{\text{max}} = 8.911$ м: $D_{\text{max}} = 2 \cdot 1.754 \cdot \sqrt{1 + \left(\frac{8.911}{0.702}\right)^2} \approx 44.67$ мм
\end{itemize}

\section{Вывод формулы для фокусного расстояния}

Требуемый диаметр пятна в фокусе: $d = 60$ мкм, что соответствует радиусу перетяжки $\omega_{02} = 30$ мкм. 
Увеличение линзы определяется как:
\[
\Gamma = \frac{\omega_{02}}{\omega_{01}} = \frac{0.030}{1.754} \approx 0.0171
\]

Найдем фокусное расстояние формулы преобразования пучка линзой:
\[
\Gamma = \frac{f}{\sqrt{(s - f)^2 + z_{R1}^2}}
\]

Решаем уравнение относительно $f$. После возведения в квадрат и преобразований получаем квадратное уравнение:
\[
(\Gamma^2 - 1)f^2 - 2\Gamma^2 s f + \Gamma^2(s^2 + z_{R1}^2) = 0
\]

Решение уравнения:
\[
f = \frac{\Gamma^2 s - \Gamma \sqrt{s^2 + (1 - \Gamma^2)z_{R1}^2}}{\Gamma^2 - 1}
\]

\section{Расчёт фокусного расстояния для различных положений}

Для минимального расстояния $s_{\text{min}} = 0.761$ м
\[
f_{\text{min}} \approx 17.48 \, \text{мм}
\]

Для центрального положения $s_{\text{center}} = 4.836$ м
\[
f_{\text{center}} \approx 82.12 \, \text{мм}
\]

Для максимального расстояния $s_{\text{max}} = 8.911$ м
\[
f_{\text{max}} \approx 150.2 \, \text{мм}
\]

\section{Анализ результатов}

\begin{itemize}
    \item Диаметр пучка на фокусирующей линзе изменяется в 8.6 раза (от 5.17 мм до 44.67 мм)
    \item Требуемое фокусное расстояние изменяется от 17.48 мм до 150.2 мм (в 8.6 раза)
    \item Невозможно обеспечить стабильный размер фокального пятна одной фокусирующей линзой
\end{itemize}

\section{Вывод о необходимости телескопической системы}

Значительное изменение диаметра пучка на фокусирующей линзе и соответствующее изменение требуемого фокусного расстояния демонстрирует необходимость использования телескопической системы для стабилизации диаметра пучка.

\section{Применение телескопической системы GBE05-E3}

Для стабилизации диаметра пучка в оптической системе используется коммерчески доступный телескоп GBE05-E3 производства Thorlabs с увеличением 5x, устанавливаемый непосредственно на выходное окно лазера~\cite{Tele}.


\begin{figure}[H]
    \centering
    \includegraphics[width=0.75\textwidth]{Tele.jpg}
    \caption{Расширитель пучка GBE05-E3}
\end{figure}

\subsection{Параметры телескопа GBE05-E3}

\begin{itemize}
    \item Увеличение: $\beta = 5$
    \item Максимальный входной диаметр пучка: 3.5 мм
    \item Просветляющее покрытие: для 10.6 мкм
    \item Типичное пропускание: >94\%
\end{itemize}

\subsection{Параметры пучка после телескопа}

После прохождения телескопа параметры пучка составляют:

\begin{align*}
    \omega_{0,\text{нов}} &= \beta \cdot \omega_{01} = 5 \cdot 1.754 = 8.77 \, \text{мм} \\
    z_{R,\text{нов}} &= \beta^2 \cdot z_{R1} = 25 \cdot 0.702 = 17.55 \, \text{м} \\
    \theta_{\text{нов}} &= \frac{\theta}{\beta} = \frac{2.5}{5} = 0.5 \, \text{мрад}
\end{align*}

\subsection{Расстояния до фокусирующей линзы после телескопа}

Поскольку телескоп установлен на выходном окне лазера, расстояние от новой перетяжки до фокусирующей линзы:

\begin{align*}
    s'_{\text{min}} &= 0.2 + 0.4 + 0 + 0 + 0 = 0.6 \, \text{м} \\
    s'_{\text{center}} &= 0.2 + 0.4 + 2.5 + 1.5 + 0.075 = 4.675 \, \text{м} \\
    s'_{\text{max}} &= 0.2 + 0.4 + 5 + 3 + 0.15 = 8.75 \, \text{м}
\end{align*}

\subsection{Стабилизация диаметра пучка}

Диаметр пучка на фокусирующей линзе после телескопа:

\begin{align*}
    D_{\text{min}} &= 2\omega_{0,\text{нов}} \sqrt{1 + \left(\frac{s'_{\text{min}}}{z_{R,\text{нов}}}\right)^2} = 17.54 \cdot \sqrt{1 + \left(\frac{0.6}{17.55}\right)^2} \approx 17.55 \, \text{мм} \\
    D_{\text{center}} &= 17.54 \cdot \sqrt{1 + \left(\frac{4.675}{17.55}\right)^2} \approx 18.15 \, \text{мм} \\
    D_{\text{max}} &= 17.54 \cdot \sqrt{1 + \left(\frac{8.75}{17.55}\right)^2} \approx 19.60 \, \text{мм}
\end{align*}

Относительное изменение диаметра: $\dfrac{19.60}{17.55} = 1.117$ раза (11.7\%)

\subsection{Требуемое фокусное расстояние}

Увеличение линзы после телескопа:
\[
\Gamma = \frac{\omega_{02}}{\omega_{0,\text{нов}}} = \frac{0.030}{8.77} \approx 0.00342
\]

Требуемое фокусное расстояние рассчитывается по формуле:
\[
f = \frac{\Gamma^2 s - \Gamma \sqrt{s^2 + (1 - \Gamma^2)z_{R,\text{нов}}^2}}{\Gamma^2 - 1}
\]

Расчет для различных положений:

\begin{itemize}
    \item Для $s'_{\text{min}} = 0.6$ м: $f_{\text{min}} \approx 60.00 \, \text{мм}$
    \item Для $s'_{\text{center}} = 4.675$ м: $f_{\text{center}} \approx 62.05 \, \text{мм}$
    \item Для $s'_{\text{max}} = 8.75$ м: $f_{\text{max}} \approx 66.90 \, \text{мм}$
\end{itemize}

Разброс фокусных расстояний: 6.90 мм

\subsection{Фокусирующая линза AR812-ZC-PX-38-63}

Для формирования пятна диаметром 60 мкм выбрана линза AR812-ZC-PX-38-63 с фокусным расстоянием $f = 63.5$ мм. 

Параметры линзы:
\begin{itemize}
    \item Тип: плосковыпуклая
    \item Материал: ZnSe (Цинк селенид)
    \item Диаметр: 38.1 мм
    \item Фокусное расстояние: 63.5 мм при 10.6 мкм
    \item Радиус кривизны: $R_1 = 89.14$ мм
    \item Просветляющее покрытие: 8-12 мкм (Ravg <0.5\%)
\end{itemize}

Расчет увеличения и радиуса перетяжки после линзы:
\[
\Gamma = \frac{f}{\sqrt{(s - f)^2 + z_{R,\text{нов}}^2}}, \quad \omega_{02} = \Gamma \cdot \omega_{0,\text{нов}}
\]

Результаты расчета диаметра фокального пятна ($d = 2\omega_{02}$):

\begin{align*}
    \Gamma_{\text{min}} &= \frac{0.0635}{\sqrt{(0.6 - 0.0635)^2 + 17.55^2}} \approx 0.003617, \\ d_{\text{min}} &= 2 \cdot 0.003617 \cdot 8.77 \approx 63.4 \, \text{мкм} \\
    \Gamma_{\text{center}} &= \frac{0.0635}{\sqrt{(4.675 - 0.0635)^2 + 17.55^2}} \approx 0.003495, \\ d_{\text{center}} &= 2 \cdot 0.003495 \cdot 8.77 \approx 61.3 \, \text{мкм} \\
    \Gamma_{\text{max}} &= \frac{0.0635}{\sqrt{(8.75 - 0.0635)^2 + 17.55^2}} \approx 0.003237, \\ d_{\text{max}} &= 2 \cdot 0.003237 \cdot 8.77 \approx 56.8 \, \text{мкм}
\end{align*}

Диаметр пятна составляет 56.8-63.4 мкм, что удовлетворяет требованиям технического задания (50-70 мкм).

\subsection{Расчет сферических аберраций для линзы AR812-ZC-PX-38-63}

Для оценки влияния сферических аберраций проведем расчет коэффициентов сферических аберраций для выбранной плосковыпуклой линзы~\cite{Lens}.

\subsubsection{Параметры линзы и системы}

\begin{itemize}
    \item Модель линзы: AR812-ZC-PX-38-63
    \item Фокусное расстояние: $f = 63.5$ мм
    \item Радиусы кривизны: $R_1 = 89.14$ мм, $R_2 = \infty$ (плоская)
    \item Диаметр: 38.1 мм
    \item Материал: ZnSe, показатель преломления $n = 2.4$ при $\lambda = 10.6$ мкм
    \item Максимальный диаметр пучка на линзе: $D_L = 19.60$ мм
    \item Расстояние до линзы: $s = 8.75$ м (наихудший случай)
\end{itemize}

\subsubsection{Расчет коэффициентов сферической аберрации}

Для плосковыпуклой линзы с выпуклой стороной к падающему пучку:
\begin{align*}
    a_k &= -\frac{R_2 + R_1}{R_2 - R_1} = -\frac{\infty + 89.14}{\infty - 89.14} = -1 \\
    b_k &= 1 - \frac{2f}{s} = 1 - \frac{2 \cdot 0.0635}{8.75} \approx 0.9855
\end{align*}

Коэффициент сферической аберрации $K^*$:
\begin{align*}
    K^* &= \frac{1}{4n(n-1)} \left[ \frac{n+2}{n-1}a_k^2 + 4(n+1)a_k b_k + (3n+2)(n-1)b_k^2 + \frac{n^3}{n-1} \right] \\
    &= \frac{1}{4 \cdot 2.4 \cdot 1.4} \left[ \frac{4.4}{1.4} \cdot 1 + 4 \cdot 3.4 \cdot (-1) \cdot 0.9855 + 9.2 \cdot 1.4 \cdot (0.9855)^2 + \frac{13.824}{1.4} \right] \\
    &\approx 0.90
\end{align*}

Уширение пятна за счет сферических аберраций:
\[
d_{ab} = \frac{K^* D_L^3}{32 f^2} = \frac{0.9 \cdot (0.01960)^3}{32 \cdot (0.0635)^2} \approx 52.2 \, \text{мкм}
\]

\subsubsection{Суммарный диаметр фокального пятна с учетом аберраций}

\begin{itemize}
    \item Дифракционный размер пятна: $d_{diff} = 56.8$ мкм (наихудший случай)
    \item Уширение от аберраций: $d_{ab} = 52.2$ мкм
    \item Суммарный диаметр пятна: $d_{total} = d_{diff} + d_{ab} = 109.3$ мкм
\end{itemize}

\subsubsection{Анализ результатов}

\begin{itemize}
    \item Требуемый диаметр пятна: 50-70 мкм
    \item Фактический диаметр с учетом аберраций: 109.3 мкм
\end{itemize}

Сферические аберрации линзы AR812-ZC-PX-38-63 приводят к двукратному уширению фокального пятна. Полученное значение 109.3 мкм превышает допустимое значение, что делает линзу не пригодной для выполнения требований технического задания. Необходимо применение асферической линзы.















\subsection{Асферическая фокусирующая линза с фокусным расстоянием 50.8 мм}

Для обеспечения минимальных сферических аберраций и точного соответствия требованиям по размеру фокального пятна выбрана асферическая линза Coherent с фокусным расстоянием $f = 50.8$ ммЁ\cite{ALens}.

\subsubsection{Параметры линзы}

\begin{itemize}
    \item Тип: асферическая
    \item Материал: ZnSe
    \item Диаметр: 25.4 мм
    \item Свободная апертура: 22.86 мм
    \item Фокусное расстояние: 50.8 мм
    \item Просветляющее покрытие: 8-12 мкм R <5\%
\end{itemize}

\subsubsection{Расчет параметров фокального пятна}

Расчет увеличения и радиуса перетяжки после линзы:
\[
\Gamma = \frac{f}{\sqrt{(s - f)^2 + z_{R,\text{нов}}^2}}, \quad \omega_{02} = \Gamma \cdot \omega_{0,\text{нов}}
\]

Результаты расчета диаметра фокального пятна ($d = 2\omega_{02}$):

\begin{align*}
    \Gamma_{\text{min}} &= \frac{0.0508}{\sqrt{(0.6 - 0.0508)^2 + 17.55^2}} \approx 0.002893, \\ d_{\text{min}} &= 2 \cdot 0.002893 \cdot 8.77 \approx 50.7 \, \text{мкм} \\
    \Gamma_{\text{center}} &= \frac{0.0508}{\sqrt{(4.675 - 0.0508)^2 + 17.55^2}} \approx 0.002798, \\ d_{\text{center}} &= 2 \cdot 0.002798 \cdot 8.77 \approx 49.1 \, \text{мкм} \\
    \Gamma_{\text{max}} &= \frac{0.0508}{\sqrt{(8.75 - 0.0508)^2 + 17.55^2}} \approx 0.002593, \\ d_{\text{max}} &= 2 \cdot 0.002593 \cdot 8.77 \approx 45.5 \, \text{мкм}
\end{align*}

Диаметр пятна составляет 45.5-50.7 мкм, что полностью удовлетворяет требованиям технического задания (50-70 мкм).

\subsubsection{Анализ сферических аберраций}

Асферическая линза с фокусным расстоянием 50.8 мм обеспечивает:
\begin{itemize}
    \item Пренебрежимо малые сферические аберрации благодаря асферическому профилю
    \item Дифракционно-ограниченное качество пучка
    \item Высокое пропускание (>95\%) благодаря просветляющему покрытию
    \item Отличную стойкость к лазерному излучению
\end{itemize}

Уширение пятна за счет сферических аберраций пренебрежимо мало:

\subsubsection{Проверка соответствия апертуры}

Максимальный диаметр пучка на линзе составляет 19.60 мм, что меньше свободной апертуры линзы (22.86 мм). Это обеспечивает отсутствие виньетирования.

\subsubsection{Сравнение с требованиями}

\begin{itemize}
    \item Требуемый диаметр пятна: 50-70 мкм
    \item Фактический диаметр с асферической линзой: 45.5-50.7 мкм
    \item Запас по качеству: отсутствие сферических аберраций
\end{itemize}

Данная конфигурация (телескоп 5x + асферическая линза f=50.8 мм) гарантирует качественную лазерную обработку материалов с соблюдением всех технических требований.
\endinput